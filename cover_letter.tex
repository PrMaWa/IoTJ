\documentclass[parskip]{scrartcl}
\usepackage[utf8]{inputenc}
\usepackage[english]{babel}
\usepackage[document]{ragged2e}
 
\begin{document}
 
\pagestyle{empty}

\begin{flushright}
    \Large
    31th of July 2018
\end{flushright}

\begin{center}
    \huge
    Cover Letter\\
    \vspace{0.4cm}
    \LARGE
    Test event generation for a fall-detection IoT system\\
    \vspace{0.4cm}
    \large
    Lorena Gutiérrez-Madroñal, Luigi La Blunda, Matthias F. Wagner, Inmaculada Medina Bulo
\end{center}
 
    \vspace{0.8cm}

    \normalsize
Dear Editor,

all authors have checked this manuscript and have agreed to submit it to Journal of Biomedical Informatics, to be considered for publication.
This manuscript is the authors' original work and has not been published, nor has been submitted simultaneously elsewhere.

The main contributions of this paper are:
the presentation of a new functionality of the tool IoT-EG \textit{(Internet of Things - Event Generator)}. IoT-TEG automatically 
generates a wide variety of events for the testing phases of any event-processing program. We have checked that the majority of 
Internet of Things (IoT) critical situations that want to be detected follow a specific pattern. The new functionality of IoT-TEG 
allows to simulate the behaviour of different event attributes in order to generate test events following a specific pattern. 
One of that IoT critical situations to detect are the falls. We introduce an analysis of the major parameter involved in a fall, 
the acceleration. The acceleration can measure the movement of the body, so its behaviour is analysed in two types of falls. 
The used IoT prototype which detects the falls has been improved, its hardware evolution as well as the obtained data is described 
and analysed. Finally, in order to detect the studied falls, event patterns of both fall types are defined.

We are convinced that the presented results are important contributions for researchers and practitioners.
IoT gained more and more attention recently, so the IoT systems have to handle a huge amount 
of information which arrives as events. These events need to be processed in real time
to make correct decisions. We find strongly necessary to test the IoT systems which will process that information. 
To test IoT systems, and their functionalities, it is necessary to use events with specific structures and values. Indeed,
the majority of the involved values in a critical situation follow a specific behaviour. The new functionality of IoT-TEG will 
help to simulate the behaviour of those critical situations to test the IoT systems in order to determine if they are
ready to be connected to the network. The analysis of real data is essential for the development of the introduced
functionality. So, the analysis of the data from a real IoT fall detection system has been crucial to define the 
behaviour of the involved events as well as to develop the new functionality of IoT-TEG.

Thank you so much for receiving our manuscript and considering it for review.
We highly appreciate your time and look forward to your response.

Kind regards,

Lorena, Luigi, Matthias, and Inmaculada.



\end{document}



